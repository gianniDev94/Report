
% ----------------------------
% Pakete - Bilder und Symbole 
% ----------------------------
\usepackage{placeins}
\usepackage{hyperref}
\usepackage{graphicx}          % \includegraphics (pdf-Version)
\usepackage{amssymb}           % AMS - Symbole
\usepackage{amsmath}          % AMS - Umgebungen + Befehle (\begin{cases}...\end{cases},\pmatrix)
\usepackage{csquotes}
% ----------------------------
% Pakete - Format und Sprache
% ----------------------------

\usepackage[paper=a4paper,left=2.5cm,right=2.5cm,top=2cm,bottom=2.5cm]{geometry} % Formatierung
\usepackage[ngerman]{babel}    % Deutsche Bezeichnungen und Trennung nach neuer Rechtschreibung
\usepackage[T1]{fontenc}       % Trennen mit Umlauten
%\usepackage[latin1]{inputenc} % Umlaute direkt im Text (ISO 8859-1 UNIX-Systeme)
\usepackage[utf8]{inputenc}   % Umlaute direkt im Text (MS-Windows)

% ----------------------------
% Pakete - Feineinstellungen
% ----------------------------
\usepackage{courier}           % Courier Schrift in verb-, verbatim- und listing-Umgebungen
%\usepackage{cmbright}         % Schriften-Gruppe im Text [cmbright, mathpazo (Palatino), times]
\usepackage{hyperref}         % Links im Inhaltsverzeichnis und \url
%\usepackage{microtype}        % Bessere Darstellung: margin + extra kerning, expansion, tracking, and spacing
\usepackage{titlesec}          % Packet zur Anpassung der Titel der Kapitel
\titleformat{\chapter}[block]{\bfseries\LARGE}{\thechapter}{2.75ex}{}[\vspace{1ex}\titlerule]

% ----------------------------
% Parameter
% ----------------------------

\linespread{1.05}              % Zeilenabstand einstellen
\setlength{\parindent}{0pt}    % Global \noindent
\setlength{\parskip}{5pt}      % Inhaltsverzeichnis + Platz zwischen Titel und Text

% ----------------------------
% Eigene Definitionen
% ----------------------------
\def\figureautorefname{Abbildung}
\def\tableautorefname{Tabelle}
\newcommand{\qed}{\ensuremath{\square} } % Statt q.e.d. ein kleines Quadrat
\newtheorem{Definition}{Definition}[section]
\newtheorem{Theorem}{Theorem}[section]
\newtheorem{Beispiel}{Beispiel}[section]
\newenvironment{Beweis}[1][Beweis]{\begin{trivlist}\item[\hskip \labelsep {\textit{#1 }}]}{\end{trivlist}
\hfill\qed}

% ----------------------------
% Quellcode-Listings
% ----------------------------

\usepackage{listings}
\lstset{
language    = python,
basicstyle  = \ttfamily, 
frame       = single,
numbers     = left, 
numberstyle = \scriptsize
}
\usepackage{float}
\newfloat{listing}{htbp}{scl}[chapter]
\floatname{listing}{Listing}


\usepackage[
backend=biber,
style=numeric,
sorting=ynt
]{biblatex} %Imports biblatex package
\addbibresource{references.bib}
% ----------------------------
